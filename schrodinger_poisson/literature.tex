\documentclass[]{scrartcl}

\usepackage[utf8]{luainputenc}
\usepackage[USenglish]{babel}

\title{A brief review of literature on conservative numerical schemes for Schrodinger-Poisson}

\begin{document}

\maketitle

\section{Method properties: conservation, accuracy, and need for algebraic solves}
It seems that the first work on schemes preserving both mass and energy is that
of Ringhofer \& Soler \cite{ringhofer2000discrete}.  They mention that there
are exactly 11 invariants for this system, and motivate the focus on mass and energy
conservation by referring to "breathing mode solutions".  They mention already that
the spatial scheme must satisfy "the discrete equivalent of integration by parts",
and show that combining this with the implicit trapezoidal method in time yields
conservation.  They consider this approach "too expensive" and replace the time
integration with a predictor-corrector approach, leading to a scheme that is only
linearly implicit.  The predictor-corrector scheme doesn't conserve energy, so
they use something very similar to relaxation to restore that, at the cost of
solving a scalar algebraic equation at each time step.
They go into detail about how
specific choices of differences/averages are needed to get conservation.

Ehrhardt \& Zisowsky \cite{ehrhardt2006fast} use the method of Ringhofer in 3D with spherical
symmetry and introduce and approach for non-reflecting boundary conditions.

Athanassoulis et. al. \cite{athanassoulis2023novel} give a scheme that is
linearly implicit and second-order in time, in an approach similar to that
of Besse for NLS.  In space they use a finite element scheme that I haven't
looked at closely but presumably is of SBP type.

Nemati et. al. \cite{nemati2025high} use 4th-order compact finite differences in space
and implicit trapezoidal (2nd-order) in time.  They also use ADI for multidimensional
solutions.  The scheme is linearly implicit.  They also test a "split step fourier difference"
method.

The introduction of the paper by Wang et. al. \cite{wang2025point} has a fairly thorough list of references
on numerical methods for SP, including many that do not conserve mass/energy.

\section{Test problems}
Ringhofer \& Soler don't do any numerical tests!

Nemati et. al. just do self-convergence tests with simple initial data, in 1/2/3D \cite{nemati2025high}.
Their examples don't seem to have any physical relevance.
Wang et. al. also test self-convergence for a simple 2D problem \cite{wang2025point}.
Athanassoulis et. al. actually give a "cosmological" example, but the initial conditions are not very clear.

\section{Boundary conditions}
Most papers work with periodic or homogeneous dirichlet boundary conditions.
Cheng et. al. use perfectly-matched layers \cite{cheng2022fourier}.
Ehrhardt et. al. formulate a \emph{transparent boundary condition} in terms of a Dirichlet-to-Neumann
operator and propose discretizations based on different assumptions about the region
outside the boundary (I haven't read those in detail).

\bibliographystyle{plain}
\bibliography{refs}

\end{document}

